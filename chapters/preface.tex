\addchap{Preface}


\begin{refsection}

This book is a revised and shortened version of my dissertation successfully defended at the University of Stuttgart on November 13, 2018. It presents an hypothesis-driven overview of the clausal syntax of German Sign Language and was written with two audiences in mind: On the one hand it addresses linguists interested in sign languages, and on the other hand it addresses cartographers. I do not assume that all sign language linguists have a background in Cartographic syntax and not all syntacticians have a background in sign language linguistics, so I have written this book in a way that no background knowledge on either topic is required.

The book consists of six chapters. Chapter \ref{chaptertheoreticalbackground} introduces the theoretical assumptions the book builds on and Chapter \ref{chapterone} gives some background on sign languages in general, German Sign Language in particular, and the elicitation methods used. The three chapters to follow are devoted to the three main clausal layers: Chapter \ref{cpchapter} discusses the structure of the CP in German Sign Language, Chapter \ref{ipsystem} discusses the IP domain, and Chapter \ref{insidevp} the categories inside the VoiceP. Finally, in Chapter \ref{chapterconclusions} I conclude the findings.

The main hypothesis defended in the present study is that scope is iconically mapped onto the body in German Sign Language---and maybe universally in all sign languages (cf. \citealt{bross2017scope}): The higher the scope of an operator, the higher the body part used for its expression will be. I will show that all higher CP categories are expressed with the eyes and eyebrows, lower CP categories find expression with the cheeks, and categories inside the IP are expressed manually only. First, these IP-internal categories take scope from left to right (i.\,e., the relevant lexical items precede the material over which they take scope). This behavior then switches to a left-to-right strategy just above the VoiceP. The categories inside the VoiceP are not expressed by adding manual signs, but by manipulating the movement path of the verb sign. 

\printbibliography[heading=subbibliography]
\end{refsection}