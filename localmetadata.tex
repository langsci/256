\author{Fabian Bross} %look no further, you can change those things right here.
\title{The clausal syntax of German Sign Language}
\subtitle{A cartographic approach}
% \BackTitle{Change your backtitle in localmetadata.tex} % Change if BackTitle is different from Title

\renewcommand{\lsSeries}{ogs} % use lowercase acronym, e.g. sidl, eotms, tgdi
\renewcommand{\lsSeriesNumber}{5}

\BackBody{This book presents a hypothesis-based description of the clausal structure of German Sign Language (DGS). The structure of the book is based on the three clausal layers CP, IP/TP, and VoiceP. The main hypothesis is that scopal height is expressed iconically in sign languages: the higher the scope of an operator, the higher the articulator used for its expression. The book was written with two audiences in mind: On the one hand it addresses linguists interested in sign languages and on the other hand it addresses cartographers.}

\dedication{\noindent\parbox{\textwidth}{\normalsize\itshape Tot linguae quot membra viro; mirabilis est ars \\ Quae facit articulos ore silente loqui.\medskip\\
\normalfont `A man has as many languages as limbs; wondrous is the art \\ Which makes fingers silently speak.'\medskip\\
(Anonymous)}
\cleardoublepage
\thispagestyle{empty}
\vfill
\begin{center}
\lsDedicationFont To my mother
\end{center}
\vfill}
%\typesetter{Change typesetter in localmetadata.tex}
\proofreader{
Amir Ghorbanpour,
Andreas Hölzl,
Aniefon Daniel,
Ivelina Stoyanova,
Jeroen van de Weijer,
Daniela Hanna-Kolbe,
Lachlan Mackenzie,
Jean Nitzke,
Simon Cozens,
Tom Bossuyt,
Vadim Kimmelman
}

\renewcommand{\lsID}{256} % contact the coordinator for the right number
\BookDOI{10.5281/zenodo.3560718}%ask coordinator for DOI
\renewcommand{\lsISBNdigital}{978-3-96110-218-1}
\renewcommand{\lsISBNhardcover}{978-3-96110-219-8}
